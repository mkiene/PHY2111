% COMPOSITE

\section{Quiz}

% INPUT

\subsection{(Problem 1) Conceptual Question 2.13}

A rock is thrown (not dropped) straight down from a bridge into the river below.

\subsubsection{Part A}

Immediately after being released, is the magnitude of the rock's acceleration greater than $g$, equal to $g$, less than $g$, or 0? Explain.
Match the words in the left column to the appropriate blanks in the sentences on the right.

~
\begin{solution}
	I chose that the magnitude of the rock's acceleration is \textbf{equal to $g$} because the rock is in free--fall. This is because, even though the rock was thrown downwards with force, the only force acting on the object after being released is gravity, or $g$.
\end{solution}


\subsubsection{Part B}

Immediately before hitting the water, is the magnitude of the rock's acceleration greater than $g$, equal to $g$, less than $g$, or 0? Explain.
Match the words in the left column to the appropriate blanks in the sentences on the right.

~
\begin{solution}
	I chose that the magnitude of the rock's acceleration is \textbf{equal to $g$} because the rock is in free--fall, for the same reason as above.
\end{solution}

\newpage

\subsection{(Problem 2) Conceptual Question 2.14}

(Figure 1) shows the velocity-versus-time graph for a moving object.

\subsubsection{Part A, B, C, D}

We can think of the slope of the velocity curve as the acceleration curve. The acceleration curve would be a straight line with a positive slope, intersecting the $x$-axis at point 2. In other words, the object has negative acceleration before it reaches point 2, and positive acceleration after it reaches point 2. The object's velocity is always positive, as the curve is a parabola above the $x$-axis.

\begin{enumerate}[label=\alph*.]
	\item At which numbered point or points is the object speeding up? \textbf{Point 3, because the acceleration of the object points in the same direction as the velocity.}
	\item At which numbered point or points is the object slowing down? \textbf{Point 1, because the acceleration of the object points in the opposite direction as the velocity.}
	\item At which numbered point or points is the object moving in the negative $x$ direction? \textbf{None, because the velocity is always positive.}
	\item At which numbered point or points is the object moving in the positive $x$ direction? \textbf{All points (1, 2, 3) because the velocity is always positive.}
\end{enumerate}

\begin{remark}
	I noticed that I accidentally selected "none" in addition to "Point 1" in Part B of this question. I meant to select only point 1. Time: 8:47pm
\end{remark}

\newpage

\subsection{(Problem 3) Problem 2.45}

The figure below shows a set of kinematic graphs for a ball rolling on a track. All segments of the track are straight lines, but some may be tilted.

Select the correct picture of the track with the ball's initial condition.

~

\begin{solution}
	Assuming that the downwards direction is positive acceleration and velocity, I selected option 2, which shows a track with a downslope, reaching a flat point, and then sloping down again. The problem also fails to mention if gravity is a force, which would change the problem. I assume that gravity is acting on the object, as it is a ball rolling along a track. Therefore, the ball could not increase in acceleration while traveling uphill, so the ball must be traveling downhill, meaning that the acceleration positive direction is downwards. This is a very vague question.
\end{solution}

\newpage

\subsection{(Problem 4) Problem 2.1 -- Enhanced -- with Hints and Feedback}

\subsubsection{Part A}

\begin{solution}

	Interval 1: Home to Lamppost

	Time: \SI{2}{minutes}

	Displacement:
	\[
		V_1 = \frac{200 - 600}{2} = -200
		.\]

	Interval 2: Lamppost to Tree

	Time: \SI{5}{minutes}

	Displacement:
	\[
		V_2 = \frac{1200 - 200}{5} = 200
		.\]
\end{solution}

\subsubsection{Part B}

\begin{solution}
	\begin{align*}
		v_{\mathrm{av}} &= \frac{\Delta x}{\Delta t} \\
		&= \frac{600}{\SI{7}{minutes}} \\
		&= \SI{85.71}{\frac{m}{min}}
		.\end{align*}
\end{solution}

\newpage

\subsection{(Problem 5) Problem 2.18 -- Enhanced -- with Hints and Feedback}

~

\begin{solution}

	Given

	Acceleration of Porsche: $a_p = \SI{3.5}{\frac{m}{s^2}}$

	Acceleration of Honda $a_h = \SI{3.0}{\frac{m}{s^2}}$

	Distance: $d=\SI{200}{m}$

	Honda's Head Start: $t_0 = 1.0s$
	\begin{align*}
		d &= \frac{1}{2}at^2 \\
		t &= \sqrt{\frac{2d}{a}} \\
		t_p &= \sqrt{\frac{2 \cdot 200}{3.5}} = \sqrt{\frac{400}{3.5}}   \\
		t_h &= \sqrt{\frac{400}{3}} + 1 \\
		\Delta t &= \left( \sqrt{\frac{400}{3}} + 1 \right) - \sqrt{\frac{400}{3.5}} \\
		&\approx \SI{1.86}{seconds}
		.\end{align*}

\end{solution}

\newpage

\subsection{(Problem 6) Problem 2.24 -- Enhanced -- with Expanded Hints}

~

\begin{solution}
	\begin{align*}
		v^2 &= u^2 + 2ad \\
		\text{Flea starts from rest, so} ~ u &= 0 \\
		v_{\mathrm{initial}} &= \sqrt{2ad} \\
		v_{\mathrm{final}} &= 0 \\
		a &= \SI{1000}{\frac{m}{s^2}} \\
		g &= \SI{-9.8}{\frac{m}{s^2}} \\
		0 &= \left( \sqrt{2ad}  \right)^2 - 2gd \\
		0 &= 2ad-2gd \\
		d &= \frac{2ad}{2g} \\
		d &= \frac{ad}{g} \\
		d &= \frac{1000 \cdot 0.40 \times 10^{-3}}{9.8} \\
		d &\approx \SI{0.0408}{m}
		.\end{align*}
\end{solution}

\newpage

\subsection{(Problem 7) Problem 2.31 -- Enhanced -- with Expanded Hints}

~

\begin{solution}
	\begin{align*}
		a(t) &= 2 + \frac{4-2}{2}t = 2+ 0.5t \\
		\Delta v &=  \int_{0}^{4} a(t) \dt = \int_{0}^{4} \left( 2 + 0.5t \right) \dt \\
		\Delta v &= \left[ 2t + 0.25t^2 \right]_{0}^{4} \\
		\Delta v &= \left( 2 \cdot 4 + 0.25 \cdot 4^2 \right) - \left( 2 \cdot 0 + 0.25 \cdot 0^2 \right) \\
		\Delta v &= \SI{12}{\frac{m}{s}} \\
		v &= v_0 + \Delta v \\
		v &= 6 + 12 \\
		v &= \SI{18}{\frac{m}{s}}
		.\end{align*}
\end{solution}

\newpage

\subsection{(Problem 8) Problem 2.33 -- Enhanced -- with Expanded Hints}

\subsubsection{Part A, B, C}

~

\begin{solution}
	\begin{align*}
		x(t) &= 5t^3+3t+5 \\
		v(t) &= 15t^2+3 \\
		a(t) &= 30t \\
		\text{Position} &= \SI{337}{m} \\
		\text{Velocity} &= \SI{243}{\frac{m}{s}} \\
		\text{Acceleration} &= \SI{120}{\frac{m}{s^2}}
		.\end{align*}
\end{solution}

\newpage

\subsection{(Problem 9) Problem 2.51}

~

\begin{solution}

	~
	Distance during Acceleration
	\begin{align*}
		u &= 0 \\
		d_1 &= \frac{1}{2} a_1 t_1^2 \\
		d_1 &= \frac{1}{2} \cdot 4.5 \cdot 7.2^2 = \SI{116.64}{m}
		.\end{align*}
	Distance during coasting
	\begin{align*}
		v_1 &= a_1 t_1 \\
		v_1 &= 4.5 \cdot 7.2 = \SI{32.4}{\frac{m}{s}} \\
		d_2 &= v_1 t_2 \\
		d_2 &= 32.4 \cdot 2.5 = \SI{81}{m}
		.\end{align*}
	Distance during Deceleration
	\begin{align*}
		v^2 &= u^2 + 2a_3 d_3 \\
		d_3 &= \frac{-u^2}{2a_3} \\
		d_3 &= \frac{32.4^2}{5.6} \\
		&= \SI{187.46}{m}
		.\end{align*}
	Total Distance
	\begin{align*}
		116.64 + 81 + 187.46 &= \SI{385.1}{m}
		.\end{align*}
\end{solution}
